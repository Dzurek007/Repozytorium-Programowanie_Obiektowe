\section{Podsumowanie}

\subsection{Osiągnięte cele}

Projekt został zrealizowany zgodnie z założeniami. System rezerwacji biletów lotniczych, umożliwiający użytkownikom dokonywanie rezerwacji, został zaprojektowany z wykorzystaniem wzorca MVC (Model-View-Controller) oraz bazy danych MySQL. Projekt zawiera kluczowe funkcjonalności, takie jak rejestracja użytkowników, logowanie, przegląd lotów, rezerwacja biletów oraz przegląd rezerwacji użytkowników. Całość działa stabilnie i zapewnia podstawowe funkcje umożliwiające łatwą obsługę rezerwacji.

\subsection{Problemy napotkane podczas realizacji}

W trakcie realizacji projektu napotkano kilka trudności:
\begin{itemize}
\item Problemy z konfiguracją bazy danych MySQL, które wymagały dostosowania połączeń oraz mapowania danych.
\item Optymalizacja interfejsu użytkownika pod kątem responsywności, co wiązało się z kilkoma iteracjami projektowymi.
\item Problemy z integracją różnych elementów systemu, co wymagało rozwiązań iteracyjnych i dodatkowych testów.
\end{itemize}

\subsection{Kierunki rozwoju}

Projekt może zostać rozwinięty o dodatkowe funkcjonalności:
\begin{itemize}
\item Integracja z systemami płatności online.
\item Rozwój aplikacji mobilnej umożliwiającej rezerwację biletów.
\item Rozbudowa raportów i analiz rezerwacji.

\end{itemize}

\subsection{Wnioski}
