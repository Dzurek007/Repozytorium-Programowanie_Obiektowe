\section{Streszczenie}
\subsection{Streszczenie w języku polskim}
System rezerwacji lotów to zaawansowana aplikacja umożliwiająca zarządzanie rezerwacjami, lotami i pasażerami. Aplikacja została zaimplementowana w języku Java z wykorzystaniem technologii Swing dla interfejsu użytkownika oraz bazy danych MySQL do przechowywania informacji o lotach, pasażerach oraz rezerwacjach. W projekcie wykorzystano klasy takie jak `FlightDAO`, `PassengerDAO`, `ReservationDAO` do obsługi danych oraz technologie opisane w \cite{w3schools} oraz \cite{sqlite_doc}. System oferuje dwie główne ścieżki interakcji: dla użytkowników (przeglądanie dostępnych lotów, rezerwacje, zarządzanie użytkownikami) oraz dla administratorów (zarządzanie lotami, pasażerami i rezerwacjami, analiza transakcji). Aplikacja zapewnia również system logowania i rejestracji użytkowników oraz funkcjonalność zarządzania bazą danych.

Kod źródłowy projektu jest dostępny w repozytorium GitHub: \url{https://github.com/Dzurek007/Repozytorium-Programowanie_Obiektowe}

\subsection{Summary in English}
The flight reservation system is an advanced application that enables the management of bookings, flights, and passengers. The application was implemented in Java using Swing for the user interface and MySQL database for storing information about flights, passengers, and reservations. The project utilizes classes such as `FlightDAO`, `PassengerDAO`, `ReservationDAO` for handling data, as well as technologies described in \cite{w3schools} and \cite{sqlite_doc}. The system offers two main interaction paths: for users (browsing available flights, making reservations, user management) and for administrators (managing flights, passengers, and reservations, transaction analysis). The application also provides a user login and registration system, as well as database management functionality.

The source code of the project is available in the GitHub repository: \url{https://github.com/Dzurek007/Repozytorium-Programowanie_Obiektowe}