\section{Przykłady dziedziczenia w systemie rezerwacji lotów}

\subsection{Klasa bazowa \texttt{Person} i klasy dziedziczące}

W systemie rezerwacji lotów, klasa \texttt{Person} pełni rolę klasy bazowej dla wszystkich osób zaangażowanych w systemie (np. pasażerów). Klasa ta zawiera ogólne dane dotyczące osoby, takie jak imię, nazwisko oraz identyfikator. Na podstawie tej klasy zostały utworzone klasy dziedziczące, takie jak \texttt{Passenger} (Pasażer) i \texttt{VIPPassenger} (Pasażer VIP).

\subsubsection{Kod klasy \texttt{Person}}

Klasa \texttt{Person} zawiera podstawowe dane osobowe, takie jak identyfikator, imię i nazwisko, które są wspólne dla różnych typów użytkowników systemu.

\begin{lstlisting}[language=Java, caption=Klasa Person]
package com.example.airlinereservation.models;

public class Person {
    protected int id;
    protected String firstName;
    protected String lastName;

    public Person(int id, String firstName, String lastName) {
        this.id = id;
        this.firstName = firstName;
        this.lastName = lastName;
    }

    public int getId() { return id; }
    public String getFirstName() { return firstName; }
    public String getLastName() { return lastName; }
}
\end{lstlisting}

\subsubsection{Kod klasy \texttt{FrequentPassenger}}

Klasa \texttt{FrequentPassenger} dziedziczy po klasie \texttt{Person}, rozszerzając ją o właściwości związane z częstymi podróżnikami, takie jak numer frequent flyer i liczba mil.

\begin{lstlisting}[language=Java, caption=Klasa FrequentPassenger]
package com.example.airlinereservation.models;

public class FrequentPassenger extends Person {
    private String frequentFlyerNumber;
    private int miles;

    public FrequentPassenger(int id, String firstName, String lastName, String frequentFlyerNumber, int miles) {
        super(id, firstName, lastName);
        this.frequentFlyerNumber = frequentFlyerNumber;
        this.miles = miles;
    }

    public String getFrequentFlyerNumber() { return frequentFlyerNumber; }
    public int getMiles() { return miles; }
}
\end{lstlisting}

\subsubsection{Kod klasy \texttt{Flight}}

Klasa \texttt{Flight} reprezentuje dane dotyczące lotu, takie jak numer lotu, miejsce startu, miejsce docelowe i data lotu. To klasa, która będzie używana w kontekście rezerwacji i przypisania pasażerów do określonych lotów.

\begin{lstlisting}[language=Java, caption=Klasa Flight]
package com.example.airlinereservation.models;

import java.time.LocalDate;

public class Flight {
    private String flightNumber;
    private String origin;
    private String destination;
    private LocalDate date;

    public Flight(String flightNumber, String origin, String destination, LocalDate date) {
        this.flightNumber = flightNumber;
        this.origin = origin;
        this.destination = destination;
        this.date = date;
    }

    public String getFlightNumber() { return flightNumber; }
    public String getOrigin() { return origin; }
    public String getDestination() { return destination; }
    public LocalDate getDate() { return date; }
}
\end{lstlisting}

\subsubsection{Kod klasy \texttt{InternationalFlight}}

Klasa \texttt{InternationalFlight} dziedziczy po klasie \texttt{Flight} i dodaje specyficzne właściwości, takie jak wymaganie paszportu dla pasażerów.

\begin{lstlisting}[language=Java, caption=Klasa InternationalFlight]
package com.example.airlinereservation.models;

public class InternationalFlight extends Flight {
    private boolean passportRequired;

    public InternationalFlight(String flightNumber, String origin, String destination, LocalDate date, boolean passportRequired) {
        super(flightNumber, origin, destination, date);
        this.passportRequired = passportRequired;
    }

    public boolean isPassportRequired() { return passportRequired; }
}
\end{lstlisting}

\subsection{Zastosowanie dziedziczenia w systemie}

System rezerwacji lotów w pełni wykorzystuje dziedziczenie, aby uprościć zarządzanie różnymi typami pasażerów oraz lotów. Klasy takie jak \texttt{FrequentPassenger} i \texttt{VIPPassenger} rozszerzają klasę bazową \texttt{Person}, aby dodać szczegółowe informacje dotyczące frequent flyer i statusu VIP, co pozwala na łatwe zarządzanie różnymi rodzajami pasażerów.

Podobnie, klasy takie jak \texttt{Flight} i \texttt{InternationalFlight} dzielą wspólne cechy, ale umożliwiają dodanie specyficznych właściwości, takich jak wymaganie paszportu, co zwiększa elastyczność systemu i pozwala na łatwiejsze dodawanie nowych typów lotów.

\subsection{Podsumowanie}

Dziedziczenie w systemie rezerwacji lotów pozwala na stworzenie przejrzystej hierarchii klas, co sprawia, że kod jest bardziej modularny, łatwiejszy do rozbudowy i utrzymania. Dzięki dziedziczeniu możliwe jest stworzenie ogólnych klas bazowych, które mogą być rozszerzane o specyficzne właściwości, co upraszcza zarządzanie systemem rezerwacji i umożliwia łatwiejszą modyfikację w przyszłości.
